\documentclass{article}

\usepackage[utf8]{inputenc}
\usepackage[T1]{fontenc}
\usepackage[francais]{babel}
\usepackage{amsmath}

\begin{document}
\section{Machine de Turing}

Deux natures de machine, par défaut déterministes.
On les décrit en deux parties:
    - dispositif "matériel" : on a une bande, aussi appelée ruban, séparée en cases numérotées. C'est la bande de lecture/écriture. Des symboles peuvent être lus ou écrits dans des cases (par défaut 1 par case). On a ensuite une unité centrale qui va controler la machine de Turing. On a un lien entre l'UC et le ruban appelé tête de lecture/écriture. Un ordinateur est une machine de Turing.
    - dispositif "logiciel" : il constitué d'un alphabet S qui est un ensemble fini de symboles, d'un ensemble fini d'états E. Ces deux ensembles sont spécifiques à la machine de Turing. On a aussi du fonction de transition delta qui permet de guider le déroulement d'un calcul sur la MT. 
        delta : E x S -> E x S x \{-1,0,1\}
	        (e,s) -> delta(e,s)=(e',s',d')

	où :\begin{itemize}
	    \item[] e = état courant de la MT
	    \item[] s = symbole contenu dans la case pointée par la tête de lecture/écriture
	    \item[] e' = nouvel état
	    \item[] s' = nouveau symbole = symbole écrit par la tête de L/E à la place de s
	    \item[] d' = déplacement de la tête de L/E avec la convention :
	    \begin{itemize}
	        \item[] si -1 alors déplacement d'une case à gauche
		\item[] si  1 alors déplacement d'une case à droite
		\item[] si  0 alors pas de déplacement
	    \end{itemize}
	\end{itemize}

Un algorithme c'est une instanciation d'une machine de Turing.

Ex : S = \{0,1,b\} où b est le blanc.

delta |    b    |    0    |    1
-----------------------
  e1  |(e2,b,-1)|(e1,0,1) |(e1,1,1)
  e2  |         |(e4,b,-1)|(e3,b,-1)
  e3  |(e5,b,1) |(e3,b,-1)|(e3,b,-1)
  e4  |(e5,1,0) |(e4,b,-1)|(e4,b,-1)
  e5  |(e5,1,0)
  e6  |(e6,0,0)
	    
Convention : à l'instant intial, la tête pointe vers 0. A l'instant initial, la machine de Turing est dans l'état initial, ici e1. (A la fin, la tête pointe de nouveau sur la case 0)

Cet exemple va être un test de parité.

Exercice : Décrire une machine de Turing modélisant la fonction f : f(n) = 2n+1

On admet qu'il est possible de modéliser les opérations suivantes à l'aide de MT :
    - les quatre opérations arithmétiques appliquée à des entiers relatifs (appartient à Z) : +, -, *, /
    - et, ou, non
    - les boucles et les structures d'aiguillages
    - comparaisons : =, /, <, <=, >, >= pour des entiers relatifs ou pour des caractères.


\section{Taille des données}
    Une instance est l'ensemble des données.

    Soit I une instance d'un certain problème.
    Soit S fixé.
    La taille de I, notée |I| est le nombre de symboles de S nécessaires pour décrire I.
    
Ex : \begin{enumerate}
     \item I ets réduite à un entier n.
        On fixe S=\{0,1,b\}
	|I| = partie\_entiere\_sup(log2(n+1))
	    ~ log2(n)

     \item I est un graphe à n sommets codé par sa matrice d'adjacence M.
        M = (mi,j)(1<=i<=n,1<=j<=n)
	avec mi,j = 1 si (i,j) existe, 0 sinon
	|I| = n\^2

     \item I est un graphe orienté G(X,A) à n sommets, complet, pondéré par :
       p: A -> N
          a -> p(a)
       G est codé par la matrice des poids P.
       P = (pi,j)(1<=i<=n,1<=j<=n)
       avec pi,j = p(i,j)
       |I| = sum(part\_ent\_sup(log2(p(i,j)+1)))
           ~ sum(log2(pi,j))
    \end{enumerate}

\section{Complexité d'un algorithme}
    Soit (P) un problème. Soit A un algorithme (=MT) résolvant (P). Soit I une instance quelconque de (P). Appelons pas toute consultation de delta. Soit $f_A(I)$ le nombre de pas effectués par A pour résoudre I. La complexité de A, $c_A$, est la fonction définie par :
        $c_A : N ->N
	      n -> c_A(n) = max(f_A(I), |I| = n)$

    Quand on ne précise pas, on exprime la complexité dans le pire des cas.

    En général, on considère le nombre d'opérations élémentaires au lieu du nombre de pas effectués par l'algorithme.
    Ex: g(x) = 2x+1 -> deux opérations élémentaires

Notation de Landau : O
    f=O(g) ou f(x) = O(g(x))
    s'il existe k>0 et N>0 tel que
        n>=N, |f(n)| <= K|g(n)|

    Ex: si f = 10000 et g = 1, f = O(g)

Définition : un algorithme est dit polynomial si sa complexité est majorable asymptotiquement par un polynôme en la taille des données. Sinon, il est dit exponentiel.
Définition : Un problème est dit polynomial s'il existe un algorithme polynomial pour le résoudre.
Ex: \begin{enumerate}
    \item Trier n entiers
    \item Primalité. Instance : un entier n (écrit sous forme binaire)
        Question : n est-il premier ?
	ATTENTION : (cas sqrt(n)) il n'est pas polynomial car on parle en taille de donnée qui est en log ! Sqrt n'est pas majorable avec un polynôme en log.
    \item Le problème suivant n'est pas connu pour être polynomial.
        Nom : Cycle hamiltonien
	Instance : un graphe G = (X,A), non orienté, codé par sa matrice d'adjacence.
	Question : existe-t-il un cycle passant exactement 1 fois par chaque sommet de G ?
    \end{enumerate}

\section{Typologie des problèmes}
Problème de décision (ou de reconnaissance) : tout problème dans lequel on pose une question dont la réponse est "oui" ou "non".
Problème d'optimisation combinatoire : problème de la forme Min/max f(x) pour x appartenant à X (supposé fini).On suppose f : X -> N (ou Z ou Q)
Problème de recherche : on cherche une solution x\up{*} d'un problème d'optimisation. Ex : f(x\up{*}=Max f(x), x appartenant à X.

Liens entre ces problèmes :
    Soit ($P_O$) un problème d'optimisation :
        Max f(x) x appartient à X, avec X finie et f : X -> N
	O lui associe un problème de décision ($P_B$) défini par :
	    Nom : $P_B$
	    Instance : X,f; une entier K >= 0 (car f : X -> N)
	    Question : Existe-t-il x0 appartenant à X tel que $f(x_0) >= K$ ?

\end{document}
